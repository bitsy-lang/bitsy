\documentclass{article}
\setlength{\parskip}{1em}

\usepackage{amsmath}

% Define typographical commands for typing judgments
\newcommand{\judgment}[2]{\ensuremath{#1 \vdash #2}}
\newcommand{\oftype}[2]{\ensuremath{#1 : #2}}

\newcommand{\word}[1]{\text{Word}\langle #1 \rangle}
\newcommand{\vect}[2]{\text{Vec}\langle #1, #2 \rangle}
\newcommand{\valid}[1]{\text{Valid}\langle #1 \rangle}
\newcommand{\idx}[1]{\lbrack #1 \rbrack}
\newcommand{\as}[0]{\text{ as }}
\newcommand{\lett}[0]{\text{let }}
\newcommand{\mux}[0]{\text{mux}}
\newcommand{\cat}[0]{\text{cat}}

\newcommand{\typingrule}[4]{
    \begin{samepage}
    \begin{flalign*}
        \text{#1} \quad &
        \frac{
            #2
        }{
            #3
        }
        &&
    \end{flalign*}

    \noindent#4
    \end{samepage}
}

\begin{document}

\title{Bitsy Type System}
\author{Michael Maloney}
\maketitle

\section{Bitsy Language}
Bitsy is a hardware description language with a principled type system founded on type theory.

\subsection{Syntax}

\begin{align*}
    &e &::= &\quad x \quad &(\text{paths}) \\
    &&| &\quad v \quad &(\text{literals}) \\
    &&| &\quad [e_0, \dots, e_{n-1}] \quad &(\text{vectors}) \\
    &&| &\quad @\text{Valid}(e) \quad &(\text{valid } e) \\
    &&| &\quad @\text{Invalid} \quad &(\text{invalid}) \\
    && \dots
\end{align*}

\begin{align*}
    &T &::= &\quad \word{n} \quad &(\text{n-bit integers}) \\
    &&| &\quad \vect{T}{n} \quad &(\text{vectors of length } n) \\
    &&| &\quad \valid{T} \quad &(\text{possibly invalid } T) \\
\end{align*}

\subsection{Typing Rules}

\typingrule{Path}{
    x : T \in \Gamma
}{
    \judgment{\Gamma}{x \uparrow T}
}{
    We can always infer the type of a path that appears in the context.
}

\typingrule{InferCheck}{
    \judgment{\Gamma}{x \uparrow T}
}{
    \judgment{\Gamma}{x \downarrow T}
}{
    If we can infer the type of an expression, we can check it.
}

\typingrule{TypeAscription}{
    \judgment{\Gamma}{e \downarrow T}
}{
    \judgment{\Gamma}{e \as T \uparrow T}
}{
    You can ascribe an expression $e$ with a type $T$ with \as.
    If you can check $e$ against $T$, then you can infer $e \as T$ against $T$.
}

\typingrule{VecCons}{
    \judgment{\Gamma}{x_i \downarrow T}
}{
    \judgment{\Gamma}{[x_0, \dots, x_{n-1}]  \downarrow T}
}{
    If you can check the type of each component $e_i$ against $T$, then you have a vector of $T$'s.
}
\typingrule{Let}{
    \judgment{\Gamma}{e \uparrow S} \quad
    \judgment{\Gamma, e : S}{b \downarrow T}
}{
    \judgment{\Gamma}{\lett x = e \{ b \} \downarrow T}
}{
    If $b$ would be well-typed as $T$ when adding a new varaible $x : S$ to the context,
    and $e$ has type $S$, then we can make these into a $\text{let}$ binding.
}

\typingrule{Add}{
    \judgment{\Gamma}{x \downarrow \word{n}} \quad
    \judgment{\Gamma}{y \downarrow \word{n}}
}{
    \judgment{\Gamma}{x + y \downarrow \word{n}}
}{
    If both $x$ and $y$ can be inferred to be $\text{Word}$ s of the same bitlength,
    then $x + y$ can be inferred to be a $\text{Word}$ of that same bitlength.
}

\typingrule{Mux}{
    \judgment{\Gamma}{c \downarrow \word{1}} \quad
    \judgment{\Gamma}{e_1 \downarrow T}\quad
    \judgment{\Gamma}{e_2 \downarrow T}\quad
}{
    \judgment{\Gamma}{\text{mux}(c, e_1, e_2) \downarrow T}
}{
    You can create a $\mux$ when $c$ is a $\word{1}$ and $e_1$ and $e_2$ have the same type.
}

\typingrule{Cat}{
    \judgment{\Gamma}{e_i \uparrow \word{m_i}} \quad
    \sum{m_i} = n
}{
    \judgment{\Gamma}{\cat(e_0, \dots, e_{n-1}) \downarrow \word{n}}
}{
    If you can infer each $e_i$ to a $\text{Word}$ and the total of their bitlengths is $n$,
    then $\cat(e_0, \dots, e_{n-1})$ is a $\word{n}$.
}

\typingrule{Sext}{
    \judgment{\Gamma}{e \uparrow \word{m}} \quad
    m \le n
}{
    \judgment{\Gamma}{\text{sext}(e) \downarrow \word{n}}
}{
    If you can infer that $e$ has type $\word{m}$ and $m \le n$,
    then you can sign extend $e$ to be $\word{n}$.
}

\typingrule{IndexWord}{
    \judgment{\Gamma}{e \uparrow \word{n}} \quad
    i < n
}{
    \judgment{\Gamma}{e \idx{i} \downarrow \word{1}}
}{
    You can statically index into a $\word{n}$ and receive a $\word{1}$.
}

\typingrule{IndexRangeWord}{
    \judgment{\Gamma}{e \uparrow \word{n}} \quad
    j \le i < n \quad m = i - j
}{
    \judgment{\Gamma}{e \idx{j..i} \downarrow \word{m}}
}{
    You can statically index into a $\word{n}$ with a range $j..i$
    and receive a $\text{Word}$ with bitlength $j - i$.

    \noindent Note that the \textit{larger} index $j$ comes \textit{first},
    since integers are written in big-endian order by hand.

    \noindent Also note that $j$ may be equal to $i$, resulting in a $\word{0}$.
}

\typingrule{IndexVec}{
    \judgment{\Gamma}{e \uparrow \vect{T}{n}} \quad
    i < n
}{
    \judgment{\Gamma}{e \idx{i} \downarrow T}
}{
    When indexing into a $\text{Vec}$, note that you receive a $T$ back and not a $\vec{T}{1}$.
}

\typingrule{IndexRangeVec}{
    \judgment{\Gamma}{e \uparrow \vect{T}{n}} \quad
    i \le j < n \quad m = j - i
}{
    \judgment{\Gamma}{e \idx{i..j} \downarrow \vect{T}{m}}
}{
    You can statically index into a $\vect{T}{n}$ with a range $i..j$
    and receive a $\text{Vec}$ with length $i - j$.

    \noindent Note that the \textit{smaller} index $j$ comes \textit{first},
    since lists are written in little-endian order by hand.

    \noindent Also note that $i$ may be equal to $j$, resulting in a $\vect{T}{0}$.
}

\end{document}
