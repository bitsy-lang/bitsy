\documentclass{article}
\usepackage{amsmath}

% Define typographical commands for typing judgments
\newcommand{\judgment}[2]{\ensuremath{#1 \vdash #2}}
\newcommand{\oftype}[2]{\ensuremath{#1 : #2}}

\newcommand{\word}[1]{\text{Word}\langle #1 \rangle}
\newcommand{\vect}[2]{\text{Vec}\langle #1, #2 \rangle}
\newcommand{\valid}[1]{\text{Valid}\langle #1 \rangle}
\newcommand{\idx}[1]{\lbrack #1 \rbrack}

\begin{document}

\title{Bitsy Type System}
\author{Michael Maloney}
\maketitle

\section{Bitsy Language}
Bitsy is a hardware description language with a principled type system founded on type theory.

\subsection{Syntax}

\begin{align*}
    &e &::= &\quad x \quad &(\text{paths}) \\
    &&| &\quad v \quad &(\text{literals}) \\
    &&| &\quad [e_0, \dots, e_{n-1}] \quad &(\text{vectors}) \\
    &&| &\quad @\text{Valid}(e) \quad &(\text{valid } e) \\
    &&| &\quad @\text{Invalid} \quad &(\text{invalid}) \\
    && \dots
\end{align*}

\begin{align*}
    &T &::= &\quad \word{n} \quad &(\text{n-bit integers}) \\
    &&| &\quad \vect{T}{n} \quad &(\text{vectors of length } n) \\
    &&| &\quad \valid{T} \quad &(\text{possibly invalid } T) \\
\end{align*}

\subsection{Typing Rules}

\begin{align*}
    \frac{
        x : T \in \Gamma
    }{
        \judgment{\Gamma}{x \uparrow T}
    } \\
    \\
    \frac{
        \judgment{\Gamma}{x_i \downarrow T}
    }{
        \judgment{\Gamma}{[x_0, \dots, x_{n-1}]  \downarrow T}
    } \\
    \\
    \frac{
        \judgment{\Gamma}{e \uparrow S} \quad
        \judgment{\Gamma, e : S}{b \downarrow T}
    }{
        \judgment{\Gamma}{\text{let} x = e \{ b \} \downarrow T}
    } \\
    \\
    \frac{
        \judgment{\Gamma}{x \downarrow \word{n}} \quad
        \judgment{\Gamma}{y \downarrow \word{n}}
    }{
        \judgment{\Gamma}{x + y \downarrow \word{n}}
    } \\
    \\
    \frac{
        \judgment{\Gamma}{c \downarrow \word{1}} \quad
        \judgment{\Gamma}{e_1 \downarrow T}\quad
        \judgment{\Gamma}{e_2 \downarrow T}\quad
    }{
        \judgment{\Gamma}{\text{mux}(c, e_1, e_2) \downarrow T}
    } \\
    \\
    \frac{
        \judgment{\Gamma}{e_i \uparrow \word{m_i}} \quad
        \sum{m_i} = n
    }{
        \judgment{\Gamma}{\text{cat}(e_0, \dots, e_{n-1}) \downarrow \word{n}}
    } \\
    \\
    \frac{
        \judgment{\Gamma}{e \uparrow \word{m}} \quad
        m \le n
    }{
        \judgment{\Gamma}{\text{sext}(e) \downarrow \word{n}}
    } \\
    \\
    \frac{
        \judgment{\Gamma}{e \uparrow \word{n}} \quad
        i < n
    }{
        \judgment{\Gamma}{e \idx{i} \downarrow \word{1}}
    } \\
    \\
    \frac{
        \judgment{\Gamma}{e \uparrow \word{n}} \quad
        j \le i < n \quad m = j - i
    }{
        \judgment{\Gamma}{e \idx{j..i} \downarrow \word{m}}
    } \\
    \\
    \frac{
        \judgment{\Gamma}{e \uparrow \vect{T}{n}} \quad
        i < n
    }{
        \judgment{\Gamma}{e \idx{i} \downarrow T}
    } \\
    \\
    \frac{
        \judgment{\Gamma}{e \uparrow \vect{T}{n}} \quad
        j \le i < n \quad m = j - i
    }{
        \judgment{\Gamma}{e \idx{j..i} \downarrow \vect{T}{m}}
    } \\
\end{align*}

\end{document}
